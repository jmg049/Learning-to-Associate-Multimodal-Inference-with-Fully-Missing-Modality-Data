Missing modality data presents a fundamental challenge in deploying multimodal machine learning systems, particularly in real-world applications where sensor failures, resource constraints, or privacy restrictions can lead to incomplete modality information. This paper introduced \textit{Cross-Modal Association Models} (C-MAMs), demonstrating that a simple, modular approach to post-hoc missing modality reconstruction can effectively maintain multimodal model performance without requiring modifications to the base architecture or training process.

We validated three key contributions through comprehensive empirical evaluation across six multimodal models, eight datasets, and four distinct modalities. First, C-MAMs significantly improved inference-time performance, achieving up to 39\%, 30\%, and 28\% improvements in AVMNIST, MM-IMDb, and IEMOCAP respectively, compared to missing modality baselines. Second, detailed statistical analysis revealed that C-MAMs can effectively reconstruct missing modality embeddings even when perfect latent space alignment is not achieved, demonstrating the robustness of multimodal models to embedding deviations. Third, we showed that a straightforward architecture using MSE loss and minimal data can produce effective reconstructions, challenging the assumption that complex generative models or contrastive learning frameworks are necessary for handling missing modalities.

\textbf{Limitations and Future Work:} While C-MAMs provide an effective solution for missing modality reconstruction, several important limitations warrant further investigation. The exponential growth in required C-MAMs with increasing modality count presents scalability challenges in high-dimensional settings. Future work should explore techniques for reducing model redundancy while maintaining reconstruction effectiveness. Additionally, the dependence on base model representations means that C-MAMs cannot overcome inherent modality imbalances in the original training. Research into more sophisticated reconstruction architectures and loss functions could address these limitations while maintaining the core benefits of simplicity and modularity.

\textbf{Broader Implications:} The success of C-MAMs in post-hoc missing modality reconstruction has significant implications for real-world multimodal systems. Their modular design and deployment flexibility make them particularly suitable for distributed learning environments such as federated learning and Internet of Things applications, where different nodes may have access to varying subsets of modalities. In healthcare settings, for instance, C-MAMs could enable robust diagnosis across hospitals with varying sensor capabilities without requiring centralized data collection or model retraining. The demonstrated effectiveness of simple reconstruction approaches also suggests that complex architectural solutions may not always be necessary for practical multimodal problems, encouraging future research to evaluate the trade-offs between model complexity and performance gains carefully.

In conclusion, this work establishes C-MAMs as a practical and effective solution for handling missing modality data in multimodal learning. By enabling flexible, post-hoc reconstruction without compromising base model integrity, C-MAMs address a critical gap in existing approaches to missing modality handling. The comprehensive empirical evaluation and statistical analysis presented in this study validate the effectiveness of C-MAMs and provide valuable insights into the relationship between embedding reconstruction quality and downstream task performance. These findings lay the groundwork for more robust and adaptable multimodal systems while highlighting important directions for future research in this critical area of machine learning.